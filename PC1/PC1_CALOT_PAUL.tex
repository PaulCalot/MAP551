\documentclass[10pt,a4paper,twocolumn]{article}
\usepackage[utf8]{inputenc}
\usepackage[francais]{babel}
\usepackage[T1]{fontenc}

\usepackage{pgfplots} % pour plot des graphs sous laTex


\usepackage{amsmath}
\usepackage{amsfonts}
\usepackage{amssymb}
\usepackage[left=1cm,right=1cm,top=1cm,bottom=1cm]{geometry}
\author{Paul Calot}
\title{MAP551 - PC 1 - Théorie de l’explosion thermique}
\begin{document}

\maketitle

%\chapter{Théorie de l'explosion thermique}

\section{Astuces}

\begin{itemize}
 \item Regarder que la solution est physiquement possible (température évolue dans le bon sens, bonne condition initiale etc.) -> attention peut être plus compliqué qu'il n'y paraît;
 \item Valider l'évolution ensuite (si on arrive à tracer) ;
 \item Tracer plan de phase (vitesse en fonction de la position, fuel en fonction de la température) => l'évolution des différentes grandeurs.
\end{itemize}

\section{Modèle simplifié 1 - explosion adiabatique}

\subsection{2.1.1}

\subsubsection{2.1.1.a}

En posant : $ H = T_r Y + T $ on obtient $d_tH = T_r \space d_tY+d_tT = 0$. 

Par conséquent:  $$\forall t \geq 0, \space H(t) = cte = H(0) = T_r Y(0) + T(0) = Tr + T_0 = T_b $$

Donc : $$\forall t \geq 0, \space H(t) = T_b$$

Puis, on a $T= H-T_rY$ d'où : 
$$ d_tY = -B e^{-\frac{E}{RT}Y} = -B e^{-\frac{E}{R(T_b-T_rY)}} Y = \Phi(Y)$$

On a également $Y= \frac{T_b-T}{T_r}$ donc : 
$$ d_tT = T_r \space B e^{-\frac{E}{RT}} (\frac{T_b-T}{T_r}) = B \space (T_b-T) e^{-\frac{E}{RT}}  = \Lambda(T)$$

\subsubsection{2.1.1.b}

On remarque que : $$ d_tT > 0 \Leftrightarrow T_b > T $$ car $B>0$ et la fonction exponentielle est toujours scrictement positive. Et $T_b = T \Leftrightarrow d_t T = 0$.

De plus et par hypothèse, $T_r > 0$ donc $T_b > T(0)$ donc $d_t T(0) > 0$. Donc $T$ est strictement croissante tant qu'elle est inférieure à $T_b$. Lorsque $T = T_b$, $d_t T = 0$ et donc $T$ n'évolue plus. Par conséquent, $T_b$ est une borne supérieure de $T$ pour une condition initiale $T(0) < T_b$ que $T$ atteindra, au pire, en un temps infini.

Donc, 
$$\lim_{t \rightarrow \infty} Y = \lim_{t \rightarrow \infty} \frac{T_b-T}{T_r} = 0 $$

\subsubsection{2.1.1.c}

On a $Y= \frac{T_b-T}{T_r}$, donc :

\begin{tikzpicture}
\begin{axis}[
    axis lines = left,
    xlabel = $T$,
    ylabel = {$Y$},
]
%Below the red parabola is defined
\addplot [
    domain=400:3000, 
    samples=1000, 
    color=red,
]
{(3000-x)/(2600)};

\end{axis}
\end{tikzpicture}


\section{Modèle simplifié 2 - explosion avec prise en compte des pertes thermiques}

On suppose que la température reste homogène et que la température des parois sont constantes (terme physique).

\section{Modèle simplifié 3 - explosion avec prise en compte de la convection}
On commence à regarder la possibilité d'avoir une échelle spatiale.

\end{document}