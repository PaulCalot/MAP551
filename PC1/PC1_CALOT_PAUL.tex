\documentclass[10pt,a4paper,twocolumn]{article}
\usepackage[utf8]{inputenc}
\usepackage[francais]{babel}
\usepackage[T1]{fontenc}
	
\usepackage{float}

\usepackage{amsmath}
\usepackage{amsfonts}
\usepackage{amssymb}
\usepackage[left=1cm,right=1cm,top=1cm,bottom=1cm]{geometry}

\usepackage{pgfplots} % pour plot des graphs sous laTex

\author{Paul Calot}
\title{MAP551 - PC 1 - Théorie de l’explosion thermique}
\begin{document}

\maketitle

%\chapter{Théorie de l'explosion thermique}

\section{Astuces}

\begin{itemize}
 \item Regarder que la solution est physiquement possible (température évolue dans le bon sens, bonne condition initiale etc.) -> attention peut être plus compliqué qu'il n'y paraît;
 \item Valider l'évolution ensuite (si on arrive à tracer) ;
 \item Tracer plan de phase (vitesse en fonction de la position, fuel en fonction de la température) => l'évolution des différentes grandeurs.
\end{itemize}

\section{Modèle simplifié 1 - explosion adiabatique}

\subsection{2.1.1}

\subsubsection{2.1.1.a}

En posant : $ H = T_r Y + T $ on obtient $d_tH = T_r \space d_tY+d_tT = 0$. 

Par conséquent:  $$\forall t \geq 0, \space H(t) = cte = H(0) = T_r Y(0) + T(0) = Tr + T_0 = T_b $$

Donc : $$\forall t \geq 0, \space H(t) = T_b$$

Puis, on a $T= H-T_rY$ d'où : 
$$ d_tY = -B e^{-\frac{E}{RT}Y} = -B e^{-\frac{E}{R(T_b-T_rY)}} Y = \Phi(Y)$$

On a également $Y= \frac{T_b-T}{T_r}$ donc : 
$$ d_tT = T_r \space B e^{-\frac{E}{RT}} (\frac{T_b-T}{T_r}) = B \space (T_b-T) e^{-\frac{E}{RT}}  = \Lambda(T)$$

\subsubsection{2.1.1.b}

On remarque que : $$ d_tT > 0 \Leftrightarrow T_b > T $$ car $B>0$ et la fonction exponentielle est toujours scrictement positive. Et $T_b = T \Leftrightarrow d_t T = 0$.

De plus et par hypothèse, $T_r > 0$ donc $T_b > T(0)$ donc $d_t T(0) > 0$. Donc $T$ est strictement croissante tant qu'elle est inférieure à $T_b$. Lorsque $T = T_b$, $d_t T = 0$ et donc $T$ n'évolue plus. Par conséquent, $T_b$ est une borne supérieure de $T$ pour une condition initiale $T(0) < T_b$ que $T$ atteindra, au pire, en un temps infini.

Donc, 
$$\lim_{t \rightarrow \infty} Y = \lim_{t \rightarrow \infty} \frac{T_b-T}{T_r} = 0 $$

\subsubsection{2.1.1.c}

On a $Y= \frac{T_b-T}{T_r}$, donc :

\begin{tikzpicture}
\begin{axis}[
    axis lines = left,
    xlabel = $T$,
    ylabel = {$Y$},
]
%Below the red parabola is defined
\addplot [
    domain=400:3000, 
    samples=1000, 
    color=red,
]
{(3000-x)/(2600)};

\end{axis}
\end{tikzpicture}

\subsubsection{2.1.1.d}

Traçons quelques graphs pour commencer : 

\begin{tikzpicture}
\begin{axis}[
    axis lines = left,
    xlabel = $T$,
    ylabel = {$\frac{\Lambda(T)}{\Lambda_{max}}$},
    legend pos=north west,
]
%Below the red parabola is defined
\addplot [
    domain=0:3000, 
    samples=1000, 
    color=red,
]
{(3000.0-x)*e^(-12000.0/x))/4.1172782209038985};
\addlegendentry{$\beta = 40$}

\addplot [
    domain=0:3000, 
    samples=1000, 
    color=blue,
]
{(3000.0-x)*e^(-40000.0/x))/0.00012499436381930621};
\addlegendentry{$\beta = 100$}
\end{axis}
\end{tikzpicture}

On a : 

\begin{table}[H]
	\begin{center}
	\begin{tabular}{|c|c|c|}
 		\hline
 		T (K) | $\beta$ & 30 & 100 \\ 
 		\hline
 		1000 & 3.0e-3 & $\approx$ 0 \\ 
 		\hline
 		2400 & 0.98 & 0.28  \\
 		\hline
	\end{tabular}
	\end{center}
 \caption{Valeur de $\Lambda(T)$ pour certaines valeurs de $T$ et de $\beta$.}
\end{table}


"Forte non linéarité dépendant de l'énergie d'activation" : plus $E_a$ augmente, plus $\beta$ augmente et plus $\Lambda$ augmente.
Cependant cette augmentation n'est pas proportionnelle à  l'augmentation de $\beta$ à température égale.

\subsection{2.1.2}
\subsubsection{2.1.2.a}

On a : $ dt\Theta = dtT \frac{1}{T_{FK}}  = \frac{\Lambda(T)}{T_{FK}}$

Puis en utilisant $ T = T_{FK} \Theta + T_0 $, on obtient :

	$$ dt\Theta = B(T_r - T_{FK} \Theta)e^{-\frac{E}{R(T_{FK}\Theta+T_0}} $$
	
\subsubsection{2.1.2.b}

Sous les hypothèses (H0) et (H1) ainsi que celle de cette question, on a :
\begin{enumerate}
	\item $T_r - T_{FK} \Theta = T_r - \frac{T_0\Theta}{\beta} = T_r*(1 + o(1))$ car $\Theta << \beta$ et par (H2).
	\item $\frac{E}{R(T_{FK}\Theta+T_0} = \frac{E}{RT_0(1+\frac{\Theta}{\beta}} = \frac{E}{RT_0} (1-\frac{\Theta}{\beta} + o(\frac{\Theta}{\beta}) $ 
\end{enumerate} 

D'où : 

$$ d_t\Theta = BT_r(1+o(1))e^{-\frac{E}{RT_0}(1-\frac{\Theta}{\beta} + o(\frac{\Theta}{\beta})} \approx B T_r e^{-\frac{E}{RT_0}} e^{\Theta} = \frac{1}{t_I}e^{\Theta} $$
avec $t_I = \frac{1}{B(T_b-T_0)e^{\frac{E}{RT_0}}}$.

En repassant en coordonnées non adimensionnée, on obtient : $e^{\Theta} = e^{\frac{T-T_0}{T_{FK}}}$ qui est donc la fonction exponentielle appliquée à une forme linéaire de $T$. Partant de $T$ qui était au dénominateur de la fraction dans l'exponentielle, on comprendre l'appellation “linéarisation de Frank-Kamenetskii”.

\subsubsection{2.1.2.c}

Notons que : $\tau = \frac{t}{t_I} \Rightarrow d\tau = \frac{dt}{t_I} $.

En utilisant (5) et le changement de variable $\tau = \frac{t}{t_I}$, on obtient : 

$$ d_{\tau}\tilde{\theta}(\tau) = t_I d_{t}\tilde{\Theta}(t) = e^{\tilde{\Theta}(t)} = e^{\tilde{\theta}(\tau)} $$

Afin de résoudre cette équation, commençons par séparer les variables : 

$ d_{\tau}\tilde{\theta}(\tau) = e^{\tilde{\theta}(\tau)} \Leftrightarrow \frac{d\tilde{\theta}(\tau)}{e^{\tilde{\theta}(\tau)} } = d\tau $

Pour résoudre cela, il suffit de remarquer que :

$ \frac{u'}{e^u} = \frac{u'}{e^{-u}} = - d(e^{-u})$

Ce qui dans notre cas, permet d'écrire :

$ d_{\tau}\tilde{\theta}(\tau) = e^{\tilde{\theta}(\tau)} \Leftrightarrow  - (e^{-\tilde{\theta}(\tau)} - 1) = t $ par intégration entre $0$ et $t$ avec $\tilde{\theta}(0) = 0$.

Ce qui donne dans le cas $\tilde{\theta}(\tau) > 0$ (ce qui est toujours vrai donc on garde l'équivalence):  

$$\tilde{\theta}(\tau) = - ln(1-\tau)$$

Graphiquement, cela donne :


\begin{tikzpicture}
\begin{axis}[
    axis lines = left,
    xlabel = {$\tau = \frac{t}{t_I}$},
    ylabel = {$\tilde{\theta}(\tau)$},
    legend pos=north west,
]
%Below the red parabola is defined
\addplot [
    domain=0:1.1, 
    samples=100, 
    color=blue,
]
{-log10(1-x)/log10(e)};
\addlegendentry{$\tilde{\theta}(\tau) = - ln(1-\tau)$}


\addplot [
    domain=0:0.1, 
    samples=100, 
    color=green,
]
{1*x};
\addlegendentry{$y = x$ (tangente à l'origine)}

\end{axis}
\end{tikzpicture}


\section{Modèle simplifié 2 - explosion avec prise en compte des pertes thermiques}

On suppose que la température reste homogène et que la température des parois sont constantes (terme physique).

\section{Modèle simplifié 3 - explosion avec prise en compte de la convection}
On commence à regarder la possibilité d'avoir une échelle spatiale.

\end{document}