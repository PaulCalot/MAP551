\documentclass[10pt,a4paper]{report}



\usepackage[utf8]{inputenc}
\usepackage[francais]{babel}	
\usepackage[T1]{fontenc}

\usepackage{float}

\usepackage{amsmath}
\usepackage{amsfonts}
\usepackage{amssymb}
%\usepackage[left=1cm,right=1cm,top=1cm,bottom=1cm]{geometry}
\usepackage[]{geometry}

\usepackage{pgfplots} % pour plot des graphs sous laTex

\usepackage{subfigure}
\usepackage{graphics}

\usepackage{tkz-tab}


\setcounter{tocdepth}{3}
\setcounter{secnumdepth}{3}
\renewcommand{\thesubsubsection}{\thesubsection.\alph{subsubsection}}


\author{Paul Calot}
\title{MAP551 - PC 1 - Théorie de l’explosion thermique}
\begin{document}

\maketitle

\tableofcontents

\chapter{Introduction}
%\section{Astuces}

\chapter{Explosion dans un réacteur homogène}

\section{Modèle simplifié 1 - explosion adiabatique}

\subsection{ }

\subsubsection{} 
% 2.1.1.a

En posant : $ H = T_r Y + T $ on obtient $d_tH = T_r \space d_tY+d_tT = 0$. 

Par conséquent:  $$\forall t \geq 0, \space H(t) = cte = H(0) = T_r Y(0) + T(0) = Tr + T_0 = T_b $$

Donc : $$\forall t \geq 0, \space H(t) = T_b$$

Puis, on a $T= H-T_rY$ d'où : 
$$ d_tY = -B e^{-\frac{E}{RT}Y} = -B e^{-\frac{E}{R(T_b-T_rY)}} Y = \Phi(Y)$$

On a également $Y= \frac{T_b-T}{T_r}$ donc : 
$$ d_tT = T_r \space B e^{-\frac{E}{RT}} (\frac{T_b-T}{T_r}) = B \space (T_b-T) e^{-\frac{E}{RT}}  = \Lambda(T)$$

\subsubsection{} % 2.1.1.b

On remarque que : $$ d_tT > 0 \Leftrightarrow T_b > T $$ car $B>0$ et la fonction exponentielle est toujours scrictement positive. Et $T_b = T \Leftrightarrow d_t T = 0$.

De plus et par hypothèse, $T_r > 0$ donc $T_b > T(0)$ donc $d_t T(0) > 0$. Donc $T$ est strictement croissante tant qu'elle est inférieure à $T_b$. Lorsque $T = T_b$, $d_t T = 0$ et donc $T$ n'évolue plus. Par conséquent, $T_b$ est une borne supérieure de $T$ pour une condition initiale $T(0) < T_b$ que $T$ atteindra, au pire, en un temps infini.

Donc, 
$$\lim_{t \rightarrow \infty} Y = \lim_{t \rightarrow \infty} \frac{T_b-T}{T_r} = 0 $$

\subsubsection{} % 2.1.1.c

On a $Y= \frac{T_b-T}{T_r}$, donc :

\begin{tikzpicture}
\begin{axis}[
    axis lines = left,
    xlabel = $T$,
    ylabel = {$Y$},
]
%Below the red parabola is defined
\addplot [
    domain=400:3000, 
    samples=1000, 
    color=red,
]
{(3000-x)/(2600)};

\end{axis}
\end{tikzpicture}

\subsubsection{}%2.1.1.d

Traçons quelques graphs pour commencer : 

\begin{tikzpicture}
\begin{axis}[
    axis lines = left,
    xlabel = $T$,
    ylabel = {$\frac{\Lambda(T)}{\Lambda_{max}}$},
    legend pos=north west,
]
%Below the red parabola is defined
\addplot [
    domain=0:3000, 
    samples=1000, 
    color=red,
]
{(3000.0-x)*e^(-12000.0/x))/4.1172782209038985};
\addlegendentry{$\beta = 40$}

\addplot [
    domain=0:3000, 
    samples=1000, 
    color=blue,
]
{(3000.0-x)*e^(-40000.0/x))/0.00012499436381930621};
\addlegendentry{$\beta = 100$}
\end{axis}
\end{tikzpicture}

On a : 

\begin{table}[H]
	\begin{center}
	\begin{tabular}{|c|c|c|}
 		\hline
 		T (K) | $\beta$ & 30 & 100 \\ 
 		\hline
 		1000 & 3.0e-3 & $\approx$ 0 \\ 
 		\hline
 		2400 & 0.98 & 0.28  \\
 		\hline
	\end{tabular}
	\end{center}
 \caption{Valeur de $\Lambda(T)$ pour certaines valeurs de $T$ et de $\beta$.}
\end{table}


"Forte non linéarité dépendant de l'énergie d'activation" : plus $E_a$ augmente, plus $\beta$ augmente et plus $\Lambda$ augmente.
Cependant cette augmentation n'est pas proportionnelle à  l'augmentation de $\beta$ à température égale.

\subsection{} % 2.1.2
\subsubsection{} % 2.1.2.a

On a : $ d_t\Theta = d_tT \frac{1}{T_{FK}}  = \frac{\Lambda(T)}{T_{FK}}$

Puis en utilisant $ T = T_{FK} \Theta + T_0 $, on obtient :

	$$ d_t\Theta = B(T_r - T_{FK} \Theta)e^{-\frac{E}{R(T_{FK}\Theta+T_0}} $$
	
\subsubsection{} % 2.1.2.b

Sous les hypothèses (H0) et (H1) ainsi que celle de cette question, on a :
\begin{enumerate}
	\item $T_r - T_{FK} \Theta = T_r - \frac{T_0\Theta}{\beta} = T_r(1 + o(1))$ car $\Theta << \beta$ et par (H2).
	\item $\frac{E}{R(T_{FK}\Theta+T_0)} = \frac{E}{RT_0(1+\frac{\Theta}{\beta})} = \frac{E}{RT_0} (1-\frac{\Theta}{\beta} + o(\frac{\Theta}{\beta}) $ 
\end{enumerate} 

D'où : 

$$ d_t\Theta = \frac{BT_r}{T_{FK}}(1+o(1))e^{-\frac{E}{RT_0}(1-\frac{\Theta}{\beta} + o(\frac{\Theta}{\beta})} \approx \frac{BT_r}{T_{FK}} e^{-\frac{E}{RT_0}} e^{\Theta} = \frac{1}{t_I}e^{\Theta} $$
avec $t_I = \frac{T_{FK}}{B(T_b-T_0)e^{\frac{E}{RT_0}}} = \frac{T_{FK}e^{\beta}}{B\bar{T_r}} $.

En repassant en coordonnées non adimensionnée, on obtient : $e^{\Theta} = e^{\frac{T-T_0}{T_{FK}}}$ qui est donc la fonction exponentielle appliquée à une forme linéaire de $T$. Partant de $T$ qui était au dénominateur de la fraction dans l'exponentielle, on comprendre l'appellation “linéarisation de Frank-Kamenetskii”.

\subsubsection{} % 2.1.2.c

Notons que : $\tau = \frac{t}{t_I} \Rightarrow d\tau = \frac{dt}{t_I} $.

En utilisant (5) et le changement de variable $\tau = \frac{t}{t_I}$, on obtient : 

$$ d_{\tau}\tilde{\theta}(\tau) = t_I d_{t}\tilde{\Theta}(t) = e^{\tilde{\Theta}(t)} = e^{\tilde{\theta}(\tau)} $$

Afin de résoudre cette équation, commençons par séparer les variables : 

$ d_{\tau}\tilde{\theta}(\tau) = e^{\tilde{\theta}(\tau)} \Leftrightarrow \frac{d\tilde{\theta}(\tau)}{e^{\tilde{\theta}(\tau)} } = d\tau $

Pour résoudre cela, il suffit de remarquer que :

$ \frac{u'}{e^u} = \frac{u'}{e^{-u}} = - d(e^{-u})$

Ce qui dans notre cas, permet d'écrire :

$ d_{\tau}\tilde{\theta}(\tau) = e^{\tilde{\theta}(\tau)} \Leftrightarrow  - (e^{-\tilde{\theta}(\tau)} - 1) = t $ par intégration entre $0$ et $t$ avec $\tilde{\theta}(0) = 0$.

Ce qui donne dans le cas $\tilde{\theta}(\tau) > 0$ (ce qui est toujours vrai donc on garde l'équivalence):  

$$\tilde{\theta}(\tau) = - ln(1-\tau)$$

Graphiquement, cela donne :


\begin{tikzpicture}
\begin{axis}[
    axis lines = left,
    xlabel = {$\tau = \frac{t}{t_I}$},
    ylabel = {$\tilde{\theta}(\tau)$},
    legend pos=north west,
]
%Below the red parabola is defined
\addplot [
    domain=0:1.1, 
    samples=100, 
    color=blue,
]
{-log10(1-x)/log10(e)};
\addlegendentry{$\tilde{\theta}(\tau) = - ln(1-\tau)$}


\addplot [
    domain=0:0.1, 
    samples=100, 
    color=green,
]
{1*x};
\addlegendentry{$y = x$ (tangente à l'origine)}

\end{axis}
\end{tikzpicture}


\section{Modèle simplifié 2 - explosion avec prise en compte des pertes thermiques}

On suppose que la température reste homogène et que la température des parois sont constantes (terme physique).

\subsection{}% 2.2.1

\subsubsection{} % 2.2.1.a

\paragraph{Minoration de T}
Supposons qu'il existe $\bar{t}$ tel que $\bar{t} = inf\{t>0, T(t) = T_0\}$.

Dans ce cas, on aurait $d_tT(\bar{t}) = T_rB e^{-\frac{E}{RT}}Y > 0$ car $Y(\bar{t}) \neq 0$. Ce qui voudrait dire que dans un voisinage de $\bar{t}$, $dt_T(t) \geq 0$ et que donc :
$\exists \tilde{t} < \bar{t}, T(t) \leq T_0$ ce qui par définition de $\bar{t}$ est exclu. 

D'où il n'existe pas de moment $t>0$ tel que $T(t) = T_0$ et donc : $\forall t >0, T(t)>T_0$.

\paragraph{Etude de $H$}

Si on pose : $H = T_r Y + T$ alors $d_tH(t) = -\frac{T-T_0}{t_p} < 0, \forall t>0$ par la démonstration précédente. Par conséquent, $H$ n'est plus constant. De plus, $H$ diminue, donc: $\forall t > 0, \space H(t)< H(t=0) = T_b$ d'après la partie précédente.

\paragraph{Majoration de T}

Par ce qui précède, on a immédiatement : $\forall t > 0, \space T(t) < T_r Y(t)+T(t) = H(t) < T_b$.
De plus, comme $T(0) = T_0 < T_b$, alors : 
$$ \forall t \geq 0, \space T(t) < T_b $$

\paragraph{Conjecture sur le comportement du système}

Initialement, $d_tT(t) \approx T_rBe^{\frac{-E}{RT}}Y$ car $T$ est très proche de $T_0$.
Par conséquent on retrouve le comportement explosif de la partie précédente. Cependant, dès lorsque $T-T_0>K$ où $K < T_r$ (car $T<T_b$) est la différence de température telle que le second terme de (8) domine le premier terme, on récupère : $dt_T = -\frac{T-T_0}{t_p} < 0$, ce qui a pour conséquence de diminuer à la température et d'empêcher l'explosion de la température. Pendant ce temps, on a toujours $d_tY(t) < 0$ et donc $Y$ diminue. Donc plus le temps passe plus le premier terme va diminuer indépendemment de $T$, avec une fois $Y\approx0$, $d_tT < 0$ jusqu'à ce que $T=T_0$. Un comportement oscillatoire amorti (avec convergence vers $T_0$ avec $T>T_0$) semble possible mais dépendra probablement des constantes utilisées.

\subsubsection{}

On peut imaginer les formes de $Y(t)$ et $T(t)$ visiblent en figure \ref{comportement_syst_2}.

\begin{figure}
	
	\begin{tikzpicture}
	\begin{axis}[
		title={Température en fonction du temps},
    	axis lines = left,
    	xlabel = {$t$},
    	ylabel = {$T(t)$},
    	ymin = 0,
    	yticklabels={0,.,.,400},
    	xticklabels={,,},
    	legend pos=north west,
	]
	\addplot [
    	domain=0:0.5, 
    	samples=100, 
    	color=blue,
	]
	{400+1000*(e^(x)-1)};
	
	\addplot [
    	domain=0.5:1.55, 
    	samples=100, 
    	color=blue,
	]
	{400+1000*(e^(0.5)-1)-400*x*(x-0.5)};
	
	\end{axis}
	\end{tikzpicture}
	
	\begin{tikzpicture}
	\begin{axis}[
		title={Fraction massique en fonction du temps},
    	axis lines = left,
    	xlabel = {$t$},
    	ylabel = {$Y(t)$},
    	xticklabels={,,},
    	legend pos=north west,
	]
	\addplot [
    	domain=0:2, 
    	samples=100, 
    	color=blue,
	]
	{e^(-4*x)};

	\end{axis}
	\end{tikzpicture}
	
	\caption{Forme de $Y$ et $T$ en fonction du temps. En pratique bien sûr, la température est $C^1$.}
	\end{figure} \label{comportement_syst_2}


	\paragraph{Adimensionnement de $Y$ et $T$}
	
	On commence par poser : $\Theta = \frac{T-T_0}{T_{FK}} \Rightarrow d_t T = T_{FK} d_t\Theta$.
	
	On a alors : $$(8) \Leftrightarrow d_t\Theta = B\frac{T_r}{T_{FK}}e^{-\frac{E}{R(T_{FK}\Theta+T_0)}}Y - \frac{\Theta}{t_p} $$
	
	Puis on pose : $ \tau = \frac{t}{t_I} \Rightarrow d\tau = \frac{dt}{t_I}$
	
	Et également : $ \tilde{Y}(\tau) = Y(t)$ et $ \tilde{\Theta}(\tau) = \Theta(t)$
	
	D'où : 
	
	$$(8) \Leftrightarrow  d_\tau\tilde{\Theta} = Bt_I\frac{T_r}{T_{FK}}e^{-\frac{E}{R(T_{FK}\tilde{\Theta}+T_0)}}\tilde{Y} - t_I \frac{\tilde{\Theta}}{t_p} $$
	
	Et également : $$(7) \Leftrightarrow d_\tau\tilde{Y} = -t_IBe^{-\frac{E}{R(T_{FK}\tilde{\Theta}+T_0)}}\tilde{Y}$$
	
	\subsection{} %2.2.2
		
	\subsubsection{}%2.2.2.a
	On a : 
	$$ d_\tau\tilde{\Theta} = Bt_I\frac{T_r}{T_{FK}}e^{-\frac{E}{R(T_{FK}\tilde{\Theta}+T_0)}}\tilde{Y} - t_I \frac{\tilde{\Theta}}{t_p} = k_1 e^{-\frac{\beta}{\beta^{-1}\tilde{\Theta}+1)}}\tilde{Y} - k_2 \tilde{\Theta}$$
	
	Or :
	$ -\frac{\beta}{\beta^{-1}\tilde{\Theta}+1)} = \Theta - 1 + o(\frac{\Theta}{\beta})$ car on a montré que $T<T_b$, donc on peut supposer que $\tilde{\Theta} << \beta$ avec par (H1) : $\beta >> 1$. Il faudra vérifier sur quel intervale cette hypothèse tient.
	
	En faisant cela, on se ramène donc à une situation analogue au modèle 1) du point de vue des calculs et on obtient :
	
	$ d_\tau\tilde{\Theta} \approx k e^{\tilde{\Theta}}\tilde{Y} - k_2 \tilde{\Theta} $ 
	
	où $ k = k_1 e^{-\beta} = Bt_I \frac{T_r}{T_{FK}} e e^{-\beta} = 1$ car on avait posé : $t_I = \frac{T_{FK}e^{\beta}}{T_rB}$.
	
	Par l'hypothèse (H2) d'une grande enthalpie de réaction, on a que $\tilde{Y}$ sera relativement constant jusqu'à ce qu'on dépasse une certaine température élevé devant $T_0$. Cela revient à supposer que pour l'intervale que l'on étudie, on a $\tilde{Y} \approx 1$.
	
	Finalement et en posant $\tilde{\Theta} = \tilde{\theta}$, on obtient :
	
	$$ d_\tau\tilde{\theta} = e^{\tilde{\theta}} - \alpha_0 \tilde{\theta} $$ 
	
	où $\alpha_0 = \frac{t_I}{t_p}$ et sur un intervale de "pas trop grande température" que l'on devra qualifier par la suite.
	
	\paragraph{Intervale temporel de validité}
	
	Plusieurs paramètres sont à l'oeuvre pour la validité du modèle. Notamment, $\tilde{\Theta} << \beta $ ce qui implique qu'on ne doit pas avoir d'explosion en température. Plus $\alpha_0$ est élevé et plus les pertes thermiques limiteront cette divergence et donc plus longtemps sera valide le modèle.
	Une forte contrainte est également appliquée à $\tilde{Y}$ qui doit demeurer quasiment constante.
	
	En utilisant le notebook et notamment la cellule du modèle de l'explosion avec prise en compte des pertes thermiques et consommation de combustible (cf. figure \ref{fig:comportement_model_alpha}), on a :
	
	\begin{enumerate}
		\item $\alpha_0 = 0$ : on retrouve le comportement du modèle précédent puisqu'on ignore les pertes thermiques. Pour ce modèle, on peut considérer que l'approximation que l'on fait est valable sur l'intervale [0,1] sec (avant l'explosion en température).
		\item $\alpha_0 = 0.1$ : on constate cette fois une décroissance de la température à partir d'un certain temps (lorsque $Y \approx 0$). Pour ce modèle, on peut considérer également que l'approximation que l'on fait est valable sur l'intervale [0,1] sec.
		\item $\alpha_0 = 2.4$ :  Pour ce modèle, on peut considérer que l'approximation que l'on fait est valable sur l'intervale [0,7] sec. En effet, les pertes thermiques retardent vraiment la montée en température qui invalide le modèle et permet ainsi de limiter l'énergie disponible pour l'accélération de la cinétique de la réaction. Ainsi, $Y$ décroit lentement sur l'intervale [0,7] sec. Au bout de sept secondes, $Y$ a environ perdu $10 \%$ de sa valeur initiale (si on fixe une tolérance à $10 \%$ qui me semble bien trop élevée pour une réaction de combustion dans un réacteur cependant).
		\item $\alpha_0 = 2.5$ : on semble ici avoir passé un palier sur $\alpha_0$ puisque la température semble rester égale à sa valeur initiale (aucun pic n'est visible contrairement à précédemment). Physiquement, toute l'énergie produite par la réaction est quasi-immédiatement dégagée via les pertes thermiques empêchant ainsi tout effet d'emballement de la réaction. Cependant, $Y$ décroit plus rapidement que précédemment (je n'explique pas ce phénomène) [REVOIR CELA].
	\end{enumerate}
	
	Notons que $\beta = 30.0$ a été utilisé pour tracer cela. Un $\beta$ plus grand permet d'augmenter $T_{max-valid}$.
	
	\begin{figure}
		\centering
		\begin{subfigure}{}
        	\includegraphics[width=7cm]{screen_capture_notebook/0_.png}
    	\end{subfigure}
		\centering  	
   		\begin{subfigure}{}
        	\includegraphics[width=7cm]{screen_capture_notebook/0_1.png}
    	\end{subfigure}
    	\centering   		

   		\begin{subfigure}{}
        	\includegraphics[width=7cm]{screen_capture_notebook/2_4.png}
    	\end{subfigure}
    	\centering   		
   		\begin{subfigure}{}
        	\includegraphics[width=7cm]{screen_capture_notebook/2_5.png}
    	\end{subfigure}
   	\caption{Comportement du modèle (7)-(8) pour différentes valeurs de $\alpha_0$, respectivement : 0, 0.1, 2.4 et 2.5. Ces valeurs ont été choisies car une modification de comportement se faisait voir.}
    \label{fig:comportement_model_alpha}
	\end{figure}
	
	\subsubsection{} %2.2.2b

	Ce diagramme est disponible sur le notebook et permet de tracer le tableau suivant :

	
	\begin{figure}
	\centering
	\begin{tikzpicture}
   		\tkzTabInit{$\tilde{\theta}_0$ / 1 , $\alpha_0 < \alpha_{cr}$ / 1,  $\alpha_0 = \alpha_{cr}$ / 1, $\alpha_0 > \alpha_{cr}$ / 1}{$0$, $\tilde{\theta_l}$, $\tilde{\theta_r}$, $+\infty$}
   		\tkzTabLine{d,h,d,h,d,h,d}
   		\tkzTabLine{,\text{CV vers },\text{$\tilde{\theta_l}$=$\tilde{\theta_r}$},\text{(instable)},d,h,d}
   		\tkzTabLine{,\text{CV vers },\text{$\tilde{\theta_l}$=$\tilde{\theta_r}$},\text{(stable)},d,h,d}
	\end{tikzpicture}
	\caption{Comportement }
	\label{fig:tableau de signe}
	\end{figure}
	
	Avec $\alpha_{cr} \approx 2.72$ d'après les graphs.
	
	Physiquement, on a :
	\begin{enumerate}
		\item $\alpha_0 < \alpha_{cr}$ : les pertes thermiques ne sont pas suffisantes pour empêcher l'emballement de la réaction et l'explosion de la température.
		\item $\alpha_0 = \alpha_{cr}$ : les pertes thermiques sont tout juste suffisantes pour empêcher l'emballement de la réaction mais le moindre apport d'énergie extérieur fait exploser la température (équilibre instable). Si $T_0$ est trop important cependant, il y a divergence.
		\item $\alpha_0 > \alpha_{cr}$ : les pertes thermiques sont suffisamment importantes pour empêcher l'emballement de la réaction et avoir un équilibre stable si la température initiale est suffisamment faible. Si tel n'est pas le cas, on peut avoir une convergence vers un équilibre instable ou une explosion.
	\end{enumerate}
	
	\subsubsection{} %2.2.2c
	Les dynamiques possibles pour l'équation (9) sont détaillés dans la question précédente. Concernant les comportements limites : 
	\begin{enumerate}
		\item $\alpha_0 \rightarrow 0$ : on retrouve le cas adiabatique de la partie précédente.
		\item $\alpha_0 \rightarrow + \infty$ : dans ce cas, la position d'équilibre stable est de plus en plus proche de $\tilde{\theta}_0 = 0$ et la position d'équilibre instable devient très très grande.
	\end{enumerate}	
	

	\subsubsection{} %2.2.2.d
	Il y a trois possibilités : 
	\begin{enumerate}
		\item Lorsque $\alpha_0 < \alpha_{cr}$,  il n'y a pas de point stationnaire.
		\item Lorsque $\alpha_0 = \alpha_{cr}$,  il n'y a un point stationnaire qui est instable.
		\item Lorsque $\alpha_0 > \alpha_{cr}$,  il y a pas deux points stationnaires $\tilde{\theta}_l$ et $\tilde{\theta}_r$ avec $\tilde{\theta}_l < \tilde{\theta}_r$. Dans ce cas, le premier est stable et le second instable.
	\end{enumerate}


\section{Modèle simplifié 3 - explosion avec prise en compte de la convection}

\subsection{} %2.3.1

Quand $\Psi = 0$, on retrouve : 
$$ d_\tau\tilde{\theta} = e^{\tilde{\theta}} - \alpha_0 \tilde{\theta} $$
soit (9), soit le modèle où l'on a $Y \approx Y(0)$ et une température homogène dans le réacteur et donc aucun phénomène convectif.

\subsection{} %2.3.2

On avait déterminé un $\alpha_{cr} \approx 2.72$ dans la partie précédente. On prend donc ici : $\alpha_0 = 3.0$ afin de ne prendre aucun risque. \\

On a également sélectionné différents $\tilde{\theta}_0$ puisqu'on avait vu que la convergence / divergence du système dépendait de cette condition initiale (cf. figure \ref{fig:tableau de signe}). On voit sur la comparaison entre $e^{\theta}$ et $q^{-}(\theta)$ que $\tilde{\theta}_{cr} \approx 1.5$.  

Ainsi, la figure \ref{fig:comportement_selon_convection_inf} présente la comparaison des deux modélisation pour $1.4 = \tilde{\theta}_0 < \tilde{\theta}_{cr} \approx 1.5$. La figure \ref{fig:comportement_selon_convection_sup} présente la comparaison des deux modélisation pour $1.6 = \tilde{\theta}_0 \geq \tilde{\theta}_{cr}$. A gauche sur la figure est la modélisation sans convection et à droite avec convection. 

\begin{figure}
		\centering
		\begin{subfigure}{}
        	\includegraphics[width=7cm]{screen_capture_notebook/with_conv_inf.png}
    	\end{subfigure}
		\centering  	
   		\begin{subfigure}{}
        	\includegraphics[width=7cm]{screen_capture_notebook/without_conv_inf.png}
    	\end{subfigure}
   	\caption{Comportement des modèles (9) (à gauche) et (11) (à droite) pour $\alpha_0 = 3.0$ et $1.4 = \tilde{\theta}_0$ < $\tilde{\theta}_{cr}$}
    \label{fig:comportement_selon_convection_inf}
	\end{figure}

\begin{figure}
		\centering
		\begin{subfigure}{}
        	\includegraphics[width=7cm]{screen_capture_notebook/with_conv_sup.png}
    	\end{subfigure}
		\centering  	
   		\begin{subfigure}{}
        	\includegraphics[width=7cm]{screen_capture_notebook/without_conv_sup.png}
    	\end{subfigure}
   	\caption{Comportement des modèles (9) (à gauche) et (11) (à droite) pour $\alpha_0 = 3.0$ et $1.6 = \tilde{\theta}_0 \geq \tilde{\theta}_{cr}$}
    \label{fig:comportement_selon_convection_sup}
	\end{figure}
	
On remarque immédiatement que la présence de la convection n'a aucune incidence lorsque la dissipation thermique au bord est suffisemment importante ($\alpha_0 = 3.0 > \alpha_{cr} \approx 2.72$) et que $\tilde{\theta}_0$ est suffisamment faible. 

Cependant, lorsque  $\tilde{\theta}_0 \geq \tilde{\theta}_{cr}$, alors on distingue une différence de comportement. En effet, alors que le modèle (9) explose (comme nous l'avons étudié précédemment), on voit que le modèle (11) convergence vers la même valeurs que précédemment (environ $0.62$ ici) \textbf{avec $\theta_c = 1.030$}. En pratique, on doit attendre $tilde{\theta}_0 \approx 2.2874$ pour avoir une explosion du modèle (11). Avant cela, la température peut augmenter (jusqu'à $\theta_0 \approx 3.2$ mais redescend toujours vers $0.62$.

$\theta_c$ permet de jouer sur la divergence du système. En effet, à un $\alpha_0$ et $\tilde{\theta}_0$ fixés (disons à $3.0$ et $2.2874$), la diminution de $\theta_c$ entraîne la convergence du système. De même, l'augmenter pour $\tilde{\theta}_0 = 2.2873$ entraîne la divergence alors qu'on avait convergence dans le cas $\theta_c = 1.030$.

La prise en compte de l'influence de la convection apporte donc de nouvelles informations puisqu'on peut maintenant se dire que $\tilde{\theta}_{cr}$ est maintenant plus élevé.

\subsection{}%2.3.3

Dans le cas $\Psi = 0$, on retrouve un contexte divergent du modèle (9) d'après le tableau \ref{fig:tableau de signe} puisque $\alpha_0 < \alpha_{cr}$.

On obtient le comportement suivante pour le modèle (11)-(12) visible en figure \ref{fig:comportement_Q2.3.3} :

\begin{figure}
		\centering
		\begin{subfigure}{}
        	\includegraphics[width=15cm]{screen_capture_notebook/Q233.png}
    	\end{subfigure}
		
   	\caption{Comportement du modèle (11)-(12) pour $\alpha_0 = 2.7$ et $\theta_c = 1.01$.}
    \label{fig:comportement_Q2.3.3}
\end{figure}

On remarque qu'un comportement périodique amorti se met en place (comportement de $\theta$). Au fur et à mesure que la température diminue, le phénomène de convection thermique se fait moins fort jusqu'au point où la convection est quasiment inexistante et que la diffusion ne suffit pas à évacuer le surplus de chaleur. La réaction se ré-emballe alors et redéclenche de puissants courants convectifs qui suffisent à empêcher l'explosion et font diminuer la température jusqu'à un nouveau minimum local plus élevé que le précédent et pour lequel la convection est également presque nulle. Petit à petit ces phénomènes se font moins forts et s'amortissent. L'amortissement est dû aux pertes thermiques qui s'effectuent au bord.

Dans le cas sans convection, la diffusion est incapable d'amener le surplus d'énergique thermique présente au coeur du réacteur aux bords suffisemment rapidement pour empêcher l'explosion de la réaction.

\subsection{} %2.3.4
\subsubsection{} %2.3.4.a

De façon générale, on remarque qu'augmenter le paramètre de bifurcation diminue l'amortissement jusqu'à provoquer l'explosion. On remarque également que lorsqu'il y a une oscillation qui se met en place, une période dure environ 30 secondes.	

\begin{enumerate}
	\item Cas (1.0, 0.01) : dans ce cas la convection est faible. On a un comportement presque parfaitement oscillant pour $\theta_c = 1.045$. Au-delà on diverge (en temps très long), en deça, il semble qu'on arrive à converger également pour des temps très longs (sauf si on diminue franchement $\theta_c$ où là la convergence est plus rapide). Je ne suis pas monter au-delà de $tend = 10000$ secondes cependant.
	\item Cas (1.0, 1.1) : dans ce cas la convection initale est forte. On a un comportement qui a cependant tendance à faire exploser la température pour  $\theta_c > 1.025$ environ.
	\item Cas (1.0, 0.5) : c'est le cas intermédiaire au deux précédents. La convergence de la température semble se faire pour $\theta_c > 1.050$. 
\end{enumerate}

Dans tous les cas, on observe pour $\theta_c$ "petit", une convergence, puis lorsqu'on augmente $\theta_c$, on commence à avoir des oscillations avant de diverger pour $\theta_c$ plus grand encore.

On remarque donc qu'un phénomène de convection intitial trop faible peut être insufissant pour empêcher l'explosion si $\theta_c$ est trop important, mais qu'un phénomène de convection trop important initialement peut également provoqué l'emballement de la réaction. Ainsi, si on traçait la courbe des valeurs de $\theta_c$ à partir desquelles il y a explosion en fonction de la convection initiale on aurait une forme de chaînette dont le minimum se situerait autour de $0.5$ (ou en tout cas entre $0.01$ et $1.1$).

On a déjà expliqué physiquement pourquoi une convection insuffisante permet une explosion de la réaction. Essayons de comprendre pourquoi une convection initiale trop importante pourrait aussi causer une explosion. On remarque que pour de telles conditions initiales, la convection diminue rapidement (sous l'effet du rééquilibrage de la température au sein du milieu) jusqu'à être particulièrement très faible (cf. figure \ref{fig:explosion_high_conv}). On se retrouve alors dans un système pour lequel la réaction n'a pas débuté (ou presque pas) et une convection très faible, analogue au cas (1.0, 0.01). 

\begin{figure}
		\centering
		\begin{subfigure}{}
        	\includegraphics[width=15cm]{screen_capture_notebook/explosion_when_convection_too_high.png}
    	\end{subfigure}
		
   	\caption{Explosion lorsque la convection initiale est trop forte. Au point le plus bas du portrait de phase pour $\Psi$, on a $\Psi_{min} \approx 0.002$.}
    \label{fig:explosion_high_conv}
\end{figure}


\subsubsection{} %2.3.4.b

Comme vu précédemment, l'explosion au sein du système peut également être dû à un régime de convection initialement trop important en étant passer (selon les valeurs initiales de la convection) par un passage en régime oscillant. Cela est en accord avec la simulation directe des équation de la dynamique des fluides.

\end{document}