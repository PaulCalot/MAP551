\documentclass[10pt,a4paper,twocolumn]{article}
\usepackage[utf8]{inputenc}
\usepackage[francais]{babel}
\usepackage[T1]{fontenc}
\usepackage{amsmath}
\usepackage{amsfonts}
\usepackage{amssymb}
\usepackage[left=1cm,right=1cm,top=1cm,bottom=1cm]{geometry}
\author{Paul Calot}
\title{MAP551 - PC 1 - Théorie de l’explosion thermique}
\begin{document}
\maketitle

\chapter{PC 1 - Théorie de l'explosion thermique}

\section{Astuces}

\begin{itemize}
 \item Regarder que la solution est physiquement possible (température évolue dans le bon sens, bonne condition initiale etc.) -> attention peut être plus compliqué qu'il n'y paraît;
 \item Valider l'évolution ensuite (si on arrive à tracer) ;
 \item Tracer plan de phase (vitesse en fonction de la position, fuel en fonction de la température) => l'évolution des différentes grandeurs.
\end{itemize}

\section{Modèle simplifié 1 - explosion adiabatique}


\section{Modèle simplifié 2 - explosion avec prise en compte des pertes thermiques}

On suppose que la température reste homogène et que la température des parois sont constantes (terme physique).

\section{Modèle simplifié 3 - explosion avec prise en compte de la convection}
On commence à regarder la possibilité d'avoir une échelle spatiale.

\end{document}